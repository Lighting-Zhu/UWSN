\phantomsection
\chapter*{毕业设计小结}
\addcontentsline{toc}{chapter}{\fHei 毕业设计小结}
这次毕业设计对我来说是一次全面的考验。从了解移动水声传感网络的概念开始,阅读相关论文了解水声MAC协议的现状,到最后可以根据指定场景的需求,设计出一个实用的MAC协议,这个从零到一的过程,让人收获颇丰。

学会使用NS2的过程非常的坎坷,在ubuntu上使用NS2的体验和在windows下使用软件的区别非常大。首先要学会tcl脚本设计仿真场景,然后要仿照已有的协议,将自己提出的协议修改到NS2框架中,最后为了获得仿真曲线,还要编写shell脚本,用awk和gnuplot等工具分析仿真出的数据文本。同时阅读NS2框架使我对C++的封装、继承、多态特性有了更深入的了解。

除了软件使用的难度外,每一次修改协议后阅读数据文本,比对每一条记录的逻辑,分析仿真效果不佳的原因,然后修改参数和流程的过程也是十分繁琐的。但是在每一次阅读数据文本的过程中,对协议的细节和整体的把握都有了很大的进步。

同时,论文撰写的过程,又助于理清思路,找到突破点。学会整理和表达在以后的科研中都是极为重要的。

经历了本次毕设,我对水声传感器网络有了一定的了解,积累了一些实际经验,对以后研究生阶段的学习目标也更加明确了。


\clearpage
\endinput