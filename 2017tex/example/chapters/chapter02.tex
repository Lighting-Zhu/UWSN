\chapter{水声传感网络MAC协议研究现状 }
\section{水声传感网络MAC协议分类 }
无线传感网络使用的MAC协议,通过信道占用方法可以分为竞争协议、非竞争协议和混合协议。竞争协议包括了ALOHA、载波侦听多址接入CSMA、基站捕获多址接入FAMA、冲突避免多址接入MACA等,非竞争协议包括了时分多址TDMA、码分多址CDMA、频分多址FDMA,混合协议包括了时空MAC(Spatial—Temporal MAC,ST-MAC)等。
\subsection{竞争协议}

\subsection{非竞争协议}

\subsection{混合协议}

\section{移动水声传感网络MAC协议研究现状}
根据移动水声网络的分布式特性和可扩展性,Francisco Salvá-Garau和Milica Stojanovic提出了多集群(Multi-CLuster)\cite{Multi-Cluster Protocol for Ad Hoc Mobile Underwater Acoustic Networks Oceans2003}MAC协议。将邻近节点分在一个集群中,在单个集群内使用统一的TDMA协议通信。不同集群间通过分配不同的扩频码来减少串扰、提高扩展性。协议分为两个阶段,首先在初始化阶段分配不同集群,然后是持续时间较长的稳定传输阶段。

考虑到移动水声网络的低速率特性,Youngtae Noh和Uichin Lee等在2014年提出了DOTS(A Propagation Delay-Aware Opportunistic MAC Protocol for Mobile Underwater Networks)协议。通过被动获得的本能信息,如邻居节点传播时延表以及预期调度时序等,实现同一信道上多数据包并行发送。\cite{DOTS: A Propagation Delay-Aware Opportunistic MAC Protocol for Mobile Underwater Networks}

对于有移动节点接入的小规模单跳传感网络,毛佳、徐元欣等提出了LTM-MAC(Location-based TDMA MAC protocol for mobile underwater networks)\cite{LTM-MAC: A location-based TDMA MAC protocol for mobile underwater networks}协议,给移动节点的接入分配了较高的优先级。对于移动节点接入的小规模多跳传感网络,提出了TMM-MAC协议(TDMA-based MAC for Multiple-hops in mobile underwater networks)。利用多跳特性,允许与移动节点互不干扰的固定节点并行发送数据,同时根据数据包长短设计不同的发送机制。

基于移动节点进行数据收集的情景,邓敏、陈惠芳等人提出了混合MAC协议\cite{A Hybrid MAC Protocol in Data-collection-oriented
	Underwater Acoustic Sensor Networks},在低负载的子网中采用基于竞争的CT-MAC(ConTention-based MAC),在高负载的子网中采用基于轮询调度的RSV-MAC(ReSerVation-based MAC)。



\endinput