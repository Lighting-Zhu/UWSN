\chapter{总结与展望}
\section {全文总结}
MAC协议是移动水声传感网络的核心部分,对于不同的应用场景和需求,MAC协议设计区别很大。本用针对移动节点向固定节点进行数据采集的场景,设计并实现了应用于单跳水声网络的MAPA-MACA-CSMA协议。主要做的工作如下:

1.介绍了水声传感网络的概念,简述了水声传感网络的发展过程、体系结构和面临的问题。

2.简单介绍了水声传感网络MAC协议的分类,总结了应用于移动水声传感网络的MAC协议研究现状。

3.重点介绍了两个竞争协议,阐明了UWALOHA和SFAMA的协议流程,建立了两个协议性能分析的模型。

4.提出了针对水声传感网络设计的MAPA-MACA-CAMA协议,介绍了协议的基本原理,重点分析了协议的三个机制,分别为移动节点接入离开机制,数据帧重传机制和基于负载变化的数据帧发送机制。

5.阐明了MAPA-MACA-CAMA协议的实现方式,探讨了节点状态转换方式并描述了协议帧的格式。对协议在不同的数据帧发送模式下的性能进行了理论分析。

6.介绍了用于网络仿真的NS2软件和为水声网络设计的Aqua-Sim平台,简要描述了水声信道的仿真模型。

7.给出了衡量协议性能的四个指标。通过仿真实验,对比MAPA-MACA-CAMA,UWALOHA和SFAMA协议的性能表现,对设计的协议进行了评价。

\section {未来工作}
根据本文的分析,提出的协议已经可以很好地应用在指定场景的移动水声网络中,但仍有不少可以改进的地方。概括为以下几个方面:

1.本文采用的对比协议都是基于竞争的协议,后续将和基于分配的协议或者混合协议比较,进一步分析协议性能。在条件允许的情况下,通过组网实验,分析协议实际效果。

2.BCT帧传递了移动节点的距离信息和负载信息,下一步将充分利用BCT帧的信息,优化固定节点的传输流程。

3.由于水声传感器节点部署在海底或海洋中,网络节点的能量更新麻烦,因此考虑到协议的能量消耗,后期可以加入睡眠唤醒机制,减少节点能耗。

\endinput